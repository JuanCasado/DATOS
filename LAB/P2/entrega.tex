\documentclass [a4paper] {article}
\usepackage[utf8]{inputenc}
\title{Ciencia de datos, práctica 2}
\author{Juan Casado Ballesteros, Samuel García Gonzalez, Iván Anaya Martín}
\usepackage{Sweave}
\begin{document}
\maketitle

\begin{abstract}

\end{abstract}

\newpage
\tableofcontents
\newpage


\section{Asociación sobre los datos de las cestas de la compra}
Hemos introducido los datos de las cestas de la compra en un fichero .txt para evitar tener que escribirlos varias veces.
Utilizamos funciones que hemos creado para automatizar la lectura del fichero y también para utilizar el algoritmo apriori.
En estas funciones solo realizamos una lectura y una ejecición del algoritmo con los parámetros que nosotros hemos elegido.
En el caso del fichero txt guardamos los elementos como listas de la compra en horizonal y los transformamos a como la función que llama a apriori espera recibirlos.
En la función que llama a apriori configuramos como queremos utilizar el algoritmo de modo que no se nos impriman los detalles de su ejecución
ni se calculen las asociaciones con conjuntos vacíos.

\subsection{Cargar los datos de un .txt}
Convierte la matriz a datos booleanos y la transpone, nos es más fácil escribir los datos en horizontal y transponer que no escribirlos ya como se espera que estén.
\begin{Schunk}
\begin{Sinput}
> readAprioriFile
\end{Sinput}
\begin{Soutput}
function(file){
  muestra<-Matrix(as.matrix(read.table(file)), sparse=T)
  muestrangCMatrix<-as(muestra,"nsparseMatrix")
  t(muestrangCMatrix)
}
\end{Soutput}
\end{Schunk}

\subsection{Llamar a apriori}
\begin{Schunk}
\begin{Sinput}
> calapriori
\end{Sinput}
\begin{Soutput}
function(matrix,soporte,confianza){
    transacciones<-as(matrix,"transactions")
    asociaciones <- apriori(transacciones, 
        parameter=list(minlen=2, support=soporte,confidence=confianza), 
        control=list(verbose=F))
    inspect(asociaciones)
}
\end{Soutput}
\end{Schunk}

\subsection{Calcular asociación}
Calculamos la asociación con soporte 0.5 y confianza 0.8 para los datos de las cestas de la compra.
\begin{Schunk}
\begin{Sinput}
> calapriori(readAprioriFile("datos1.txt"),0.5,0.8)
\end{Sinput}
\begin{Soutput}
    lhs             rhs     support   confidence lift count
[1] {Agua}       => {Pan}   0.6666667 1.0        1.20 4    
[2] {Pan}        => {Agua}  0.6666667 0.8        1.20 4    
[3] {Leche}      => {Pan}   0.6666667 0.8        0.96 4    
[4] {Pan}        => {Leche} 0.6666667 0.8        0.96 4    
[5] {Agua,Leche} => {Pan}   0.5000000 1.0        1.20 3    
\end{Soutput}
\end{Schunk}

\section{Asociación sobre los datos de los vehículos}
Ya habíamos creado las función para leer datos de un .txt y suministrárselos a apriori.
Repetimos el proceso ahora con los datos de los automóbiles iobteniendo los siguentes resultados para un soporte de 0.4 y una confianza de 0.9.
\begin{Schunk}
\begin{Sinput}
> calapriori(readAprioriFile("datos2.txt"),0.4,0.9)
\end{Sinput}
\begin{Soutput}
    lhs                                 rhs              support confidence lift     count
[1] {Control_de_Velocidad}           => {Faros_de_Xenon} 0.625   1          1.333333 5    
[2] {Bluetooth}                      => {Faros_de_Xenon} 0.625   1          1.333333 5    
[3] {Bluetooth,Control_de_Velocidad} => {Faros_de_Xenon} 0.500   1          1.333333 4    
\end{Soutput}
\end{Schunk}

\section{Creación de un algoritmo apriori}
Hemos programado una versión simplificada del algoritmo eliminando algunas de las optimizaciones que este realiza.
Para implementarlo lo hemos hecho utilizando tres funciones.

Repetimos el cálculo de la asociación para los datos de las cestas de la compra y de los automóbiles comprobando que nuestro algoritmo proporciona
los mismos resultados que apriori nos había dado anteriormente. Comprobamos que el algoritmo se comporta como esperábamos.
\begin{Schunk}
\begin{Sinput}
> print(toTable(f_apriori(readAprioriFile("datos1.txt"),0.5,0.8)),right=F)
\end{Sinput}
\begin{Soutput}
  lhs             rhs     support   confidence lift count
1 {Pan}        => {Agua}  0.6666667 0.8        1.20 4    
2 {Agua}       => {Pan}   0.6666667 1.0        1.20 4    
3 {Pan}        => {Leche} 0.6666667 0.8        0.96 4    
4 {Leche}      => {Pan}   0.6666667 0.8        0.96 4    
5 {Agua,Leche} => {Pan}   0.5000000 1.0        1.20 3    
\end{Soutput}
\begin{Sinput}
> print(toTable(f_apriori(readAprioriFile("datos2.txt"),0.4,0.9)),right=F)
\end{Sinput}
\begin{Soutput}
  lhs                                 rhs              support confidence lift     count
1 {Bluetooth}                      => {Faros_de_Xenon} 0.625   1          1.333333 5    
2 {Control_de_Velocidad}           => {Faros_de_Xenon} 0.625   1          1.333333 5    
3 {Bluetooth,Control_de_Velocidad} => {Faros_de_Xenon} 0.500   1          1.333333 4    
\end{Soutput}
\end{Schunk}


\end{document}
